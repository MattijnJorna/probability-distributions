\begin{exercise}
    \textbf{[BH.12]} Let $X_1, \ldots, X_n \sim \Norm{c, \sigma^2}$ and they are \iid r.v.s. $\bar{X}_n = \frac{1}{n} \sum_{j = 1}^n X_j$, $T_1 = \bar{X}_n^2$ and $T_2 = \frac{1}{n} \sum_{j = 1}^n X_j^2$. Note that $T_1$ is the square of the first sample moment and $T_2$ is the second sample moment.
    \begin{enumerate}
\item First answer the original questions in your own words.
\item What is the distribution of $\bar{X}_n$ and what happens to the distribution when $n\rightarrow \infty$? Can you infer what would happen to the distribution of $T_1$?
\item What is the variance of $T_2$ and what happens to its distribution when  $n\rightarrow \infty$.
    \end{enumerate}
    The following codes may give you some insight:
    \begin{minted}[]{R}
f <- function(n,c, sigma, seeds){
    set.seed(seeds)
    X=rnorm(n,c,sigma)
    T1=(mean(X))^2;
    T2=mean(X^2)
    return(c(T1,T2))
}
g <- function(n,seeds){
    c=3
    sigma2=1
    sigma=sqrt(sigma2)
    return(f(n,c, sigma, seeds))
}
repeat_draw=100
T12_fixed_n<-function(n,repeat_draw){
    sapply(seq(1, repeat_draw, by=1), function(x){g(n,x)})
}
T12 <-T12_fixed_n(100,repeat_draw)
T1=T12[1,]
T2=T12[2,]
plot(T1,T2,  xlim=c(5,12),ylim=c(5,12) )
abline(0,1)

large_n=1000;
T12 <-sapply(seq(1, large_n, by=1), function(x){T12_fixed_n(x,1)})
T1=T12[1,]
T2=T12[2,]
plot(T1, type="l", lty=2,col="red", lwd=3)
lines(T2, lwd=3)
abline(h=9, col="blue")
    \end{minted}
\begin{minted}{python}
import numpy as np
from statistics import mean
from math import sqrt
import matplotlib.pyplot as plt


def f(n, c, sigma, seeds):
    np.random.seed(seeds)
    X = np.random.normal(c, sigma, n)
    T1 = (mean(X)) ** 2
    T2 = mean(X**2)
    return [T1, T2]


def g(n, seeds):
    c = 3
    sigma2 = 1
    sigma = sqrt(sigma2)
    return f(n, c, sigma, seeds)


repeat_draw = 100


def T12_fixed_n(n, repeat_draw):
    if repeat_draw != 1:
        return [g(n, x) for x in range(1, repeat_draw + 1)]
    else:
        return g(n, 1)


T12 = T12_fixed_n(100, repeat_draw)
T1 = [T12[k][0] for k in range(0, len(T12))]
T2 = [T12[k][1] for k in range(0, len(T12))]
plt.ylim(5, 12)
plt.xlim(5, 12)
plt.scatter(T1, T2)
x = list(range(5, 13))
plt.plot(x, x, "-")

large_n = 1000
T12 = [T12_fixed_n(x, 1) for x in range(1, large_n + 1)]
T1 = [T12[k][0] for k in range(0, len(T12))]
T2 = [T12[k][1] for k in range(0, len(T12))]
plt.clf()
plt.plot(T1, color="r")
plt.plot(T2, color="k")
plt.axhline(y=9, color="b", linestyle="-")
\end{minted}

\end{exercise}